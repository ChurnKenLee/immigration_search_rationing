\documentclass[12pt, letter]{article}
%%%%%%%%%%%%%%%%%%%%%%%%%%%%%%%%%%%%%%%%%%%%%%%%%%%%%%%%%%%%%%%%%%%%%%%%%%%%%%%%%%%%%%%%%%%%%%%%%%%%%%%%%%%%%%%%%%%%%%%%%%%%%%%%%%%%%%%%%%%%%%%%%%%%%%%%%%%%%%%%%%%%%%%%%%%%%%%%%%%%%%%%%%%%%%%%%%%%%%%%%%%%%%%%%%%%%%%%%%%%%%%%%%%%%%%%%%%%%%%%%%%%%%%%%%%%
\usepackage{amsmath}
\usepackage{natbib}
\usepackage{setspace}
\usepackage[labelfont={sc,singlespacing,footnotesize},textfont={sc,singlespacing,footnotesize},labelsep=colon,skip=-10pt]{caption}
\usepackage[dvipsnames]{xcolor}
\usepackage[colorlinks=true,citecolor=Blue, urlcolor=OliveGreen]{hyperref}
\usepackage[margin=0.85in]{geometry}
\usepackage{graphicx}
\usepackage{longtable}
\usepackage{float}
\usepackage{pdflscape}
\usepackage{url}
\usepackage{multibib}
\usepackage[normalem]{ulem}
\usepackage{comment}
\usepackage[flushleft]{threeparttable}
\usepackage{booktabs}
\usepackage{subcaption}


\usepackage[latin1]{inputenc}
\usepackage{amsfonts}
\usepackage{amssymb}
\usepackage{theorem}
\usepackage{color}
\usepackage{listings}
\usepackage{rotating}
\usepackage{placeins}
\usepackage{morefloats}
\usepackage{tabularx}
\usepackage{varioref}
\usepackage{epstopdf}
\usepackage{multirow}

\begin{document} 

\begin{center}
{\Large {\textbf{Replication Package} for } \bigskip }

{\Large {``\textbf{Migrants, Ancestors, and Foreign Investments}''} \bigskip }

{\Large {published in The Review of Economic Studies.}\bigskip }

\end{center}

The replication package is contained in the folder ``Replication Files" and consists of three parts:
\begin{itemize}
\item \textbf{Do Files:} There are two do files, Replication.do and Do - HDFE.do, both of which are saved in the Code folder. Replication.do is the file that generates tables and figures from the paper and online appendix while DO - HDFE.do is simply called within the Replication.do file. To use the replication files:
\begin{enumerate}
\item Save the Replication Files folder on your computer and then open the Replication.do file and enter the directory in which the folder is saved in line 56 (see line 55 for an example). 
\item Run lines 50 through 111 of the Replication.do file.
\item Find the section of code associated with the figure/table you would like to replicate and run that section of code. It is also possible to run the entire do file, though this takes a signficant amount of time.
\item Find all files generated by the code in the Output folder.
\end{enumerate}
\item \textbf{Input Data Files: } Located in the Input folder, these are all the data files that are called in running the Replication.do file.
\item \textbf{Output Data Files and Tables/Figures: } Located in the Output folder, these are all the data files that are generated in running the Replication.do file.
\end{itemize}
\end{document} 